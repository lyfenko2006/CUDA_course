\documentclass[a4paper,12pt,titlepage,draft]{article}
% при подготовке финальной версии отчёта смените опцию draft на final

\usepackage[T1,T2A]{fontenc}     % форматы шрифтов
\usepackage[utf8x]{inputenc}     % кодировка символов, используемая в данном файле
\usepackage[russian]{babel}      % пакет русификации
\usepackage{tikz}                % для создания иллюстраций
\usepackage{pgfplots}            % для вывода графиков функций
\usepackage{geometry}		     % для настройки размера полей
\usepackage{indentfirst}         % для отступа в первом абзаце секции

% выбираем размер листа А4, все поля ставим по 3см
\geometry{a4paper,left=30mm,top=30mm,bottom=30mm,right=30mm}

\begin{document}

\begin{titlepage}
    \begin{center}
	{\Large \sc Отчет по заданию}\\
	~\\
	{\large \bf <<Реализация алгоритма 3D Якоби с использованием графических процессоров>>}\\ 
    \end{center}
    \begin{flushright}
	\vfill {Выполнил:\\
	студент 201 группы\\
	Лыфенко~А.~И.\\.}
    \end{flushright}
    \begin{center}
	\vfill
	{\small Москва\\ \the\year{}}
    \end{center}
\end{titlepage}

\section{Постановка задачи}
1. Реализовать параллельный алгоритм 3-х мерного Якоби по данному последовательному алгоритму.\\

2. Оценить ускорение программы по отношению к последовательной версии.\\
\newpage

\section{Результаты работы программы}
\begin{minipage}{.5\textwidth}
\textbf{Последовательное выполнение:}\\

 Jacobi3D Benchmark Completed.\\
 Size            =  100 x  100 x  100\\
 Iterations      =                100\\
 Time in seconds =               4.80\\
 Operation type  =     floating point\\
 Verification    =         SUCCESSFUL\\
 END OF Jacobi3D Benchmark\\
 
Jacobi3D Benchmark Completed.\\
Size            =  384 x  384 x  384\\
Iterations      =                100\\
Time in seconds =             283.73\\
Operation type  =     floating point\\
Verification    =       UNSUCCESSFUL\\
END OF Jacobi3D Benchmark\\

Jacobi3D Benchmark Completed.\\
 Size            =  500 x  500 x  500\\
 Iterations      =                100\\
 Time in seconds =             630.34\\
 Operation type  =     floating point\\
 Verification    =       UNSUCCESSFUL\\
 END OF Jacobi3D Benchmark\\

Jacobi3D Benchmark Completed.\\
 Size            =  500 x  500 x  500\\
 Iterations      =                200\\
 Time in seconds =            1268.34\\
 Operation type  =     floating point\\
 Verification    =         SUCCESSFUL\\
 END OF Jacobi3D Benchmark\\
 
\end{minipage}
\begin{minipage}{.45\textwidth}
\textbf{Параллельное выполнение:}\\

 Jacobi3D Benchmark Completed.\\
 Size            =  100 x  100 x  100\\
 Iterations      =                100\\
 Time in seconds =               0.19\\
 Operation type  =     floating point\\
 Verification    =         SUCCESSFUL\\
 END OF Jacobi3D Benchmark\\

Jacobi3D Benchmark Completed.\\
 Size            =  384 x  384 x  384\\
 Iterations      =                100\\
 Time in seconds =               1.07\\
 Operation type  =     floating point\\
 Verification    =       UNSUCCESSFUL\\
 END OF Jacobi3D Benchmark\\

 Jacobi3D Benchmark Completed.\\
 Size            =  500 x  500 x  500\\
 Iterations      =                100\\
 Time in seconds =               2.21\\
 Operation type  =     floating point\\
 Verification    =       UNSUCCESSFUL\\
 END OF Jacobi3D Benchmark\\

 Jacobi3D Benchmark Completed.\\
 Size            =  500 x  500 x  500\\
 Iterations      =                200\\
 Time in seconds =               4.41\\
 Operation type  =     floating point\\
 Verification    =         SUCCESSFUL\\
 END OF Jacobi3D Benchmark\\

\end{minipage}

\newpage
\section{Сравнение скорости работы алгоритмов}
\textbf{Последовательный алгоритм:}\\

Теоритическая оценка: $O(L^3 * I)$\\

\textbf{Параллельный алгоритм:}\\

Теоритическая оценка: $O(I * ???)$
\end{document}
