\documentclass[a4paper,12pt,titlepage,draft]{article}
% при подготовке финальной версии отчёта смените опцию draft на final

\usepackage[T1,T2A]{fontenc}     % форматы шрифтов
\usepackage[utf8x]{inputenc}     % кодировка символов, используемая в данном файле
\usepackage[russian]{babel}      % пакет русификации
\usepackage{tikz}                % для создания иллюстраций
\usepackage{pgfplots}            % для вывода графиков функций
\usepackage{geometry}		     % для настройки размера полей
\usepackage{indentfirst}         % для отступа в первом абзаце секции
\usepackage{amsmath}
% выбираем размер листа А4, все поля ставим по 3см
\geometry{a4paper,left=30mm,top=30mm,bottom=30mm,right=30mm}

\begin{document}

\begin{titlepage}
    \begin{center}
	{\Large \sc Отчет по заданию}\\
	~\\
	{\large \bf <<Реализация алгоритма 3D ADI с использованием графических процессоров>>}\\ 
    \end{center}
    \begin{flushright}
	\vfill {Выполнил:\\
	студент 201 группы\\
	Лыфенко~А.~И.\\.}
    \end{flushright}
    \begin{center}
	\vfill
	{\small Москва\\ \the\year{}}
    \end{center}
\end{titlepage}

\section{Постановка задачи}
1. Реализовать параллельный алгоритм 3-х мерного ADI по данному последовательному алгоритму.\\

2. Оценить ускорение программы по отношению к последовательной версии.\\
\newpage
\section{Описание программы}
\textbf{При распараллеливании программы было создано 6 ядер:}\\
\begin{verbatim}
__global__ void init_parallel(double *a)
\end{verbatim}
Параллельно инициализирует массив.\\
\begin{verbatim}
__global__ void f1(double *a, int ii)
\end{verbatim}
Используется для счета алгоритма.
\begin{verbatim}
__global__ void f2(double *a, int jj)
\end{verbatim}
Используется для счета алгоритма.
\begin{verbatim}
__global__ void f3(double *a, int kk)
\end{verbatim}
Используется для счета алгоритма.\\
\begin{verbatim}
__global__ void f_cp(double *a, double *tmp1)
\end{verbatim}
Копирование массива.\\
\begin{verbatim}
__global__ void f4(double *a, double *tmp1, double *tmp2)
\end{verbatim}
Для нахождения eps.\\
\begin{verbatim}
__global__ void f_cp_k_i_j(double *a, double *tmp3)
\end{verbatim}
Переупорядочивает данные в массиве на k, i, j.\\
\begin{verbatim}
__global__ void f_cp_j_k_i(double *a, double *tmp3)
\end{verbatim}
Переупорядочивает данные в массиве на j, k, i.\\

\textbf{Список функций:}\\
\begin{verbatim}
double adi_parallel(double* a)
\end{verbatim}
Запускает параллельный счет алгоритма. Возвращает полученное eps.
\begin{verbatim}
void init_seq(double *a)
\end{verbatim}
Последовательно инициализирует массив.
\begin{verbatim}
double adi_seq(double* a)
\end{verbatim}
Запускает последовательный счет алгоритма. Возвращает полученное eps.
\begin{verbatim}
void print_benchmark(struct timeval startt, struct timeval endt)
\end{verbatim}
Выводит результат теста.
\newpage
\section{Результаты работы программы на различных входных данных}
\begin{minipage}{.5\textwidth}
\textbf{Последовательное выполнение:}\\

 ADI Benchmark Completed.\\
 Size            =  100 x  100 x  100\\
 Iterations      =                100\\
 Time in seconds =               0.38\\
 Operation type  =   double precision\\
 Verification    =         SUCCESSFUL\\
 END OF ADI Benchmark\\
 
 ADI Benchmark Completed.\\
 Size            =  384 x  384 x  384\\
 Iterations      =                100\\
 Time in seconds =              29.74\\
 Operation type  =   double precision\\
 Verification    =         SUCCESSFUL\\
 END OF ADI Benchmark\\

 ADI Benchmark Completed.\\
 Size            =  500 x  500 x  500\\
 Iterations      =                100\\
 Time in seconds =              66.37\\
 Operation type  =   double precision\\
 Verification    =         SUCCESSFUL\\
 END OF ADI Benchmark\\

 ADI Benchmark Completed.\\
 Size            =  500 x  500 x  500\\
 Iterations      =                200\\
 Time in seconds =             132.80\\
 Operation type  =   double precision\\
 Verification    =         SUCCESSFUL\\
 END OF ADI Benchmark\\

 ADI Benchmark Completed.\\
 Size            =  100 x  200 x  300\\
 Iterations      =                100\\
 Time in seconds =               3.04\\
 Operation type  =   double precision\\
 Verification    =         SUCCESSFUL\\
 END OF ADI Benchmark
\end{minipage}
\begin{minipage}{.45\textwidth}
\textbf{Параллельное выполнение:}\\

 ADI Benchmark Completed.\\
 Size            =  100 x  100 x  100\\
 Iterations      =                100\\
 Time in seconds =               0.32\\
 Operation type  =   double precision\\
 Verification    =         SUCCESSFUL\\
 END OF ADI Benchmark\\

 ADI Benchmark Completed.\\
 Size            =  384 x  384 x  384\\
 Iterations      =                100\\
 Time in seconds =               3.73\\
 Operation type  =   double precision\\
 Verification    =         SUCCESSFUL\\
 END OF ADI Benchmark\\

 ADI Benchmark Completed.\\
 Size            =  500 x  500 x  500\\
 Iterations      =                100\\
 Time in seconds =               8.40\\
 Operation type  =   double precision\\
 Verification    =         SUCCESSFUL\\
 END OF ADI Benchmark\\

 ADI Benchmark Completed.\\
 Size            =  500 x  500 x  500\\
 Iterations      =                200\\
 Time in seconds =             16.98\\
 Operation type  =   double precision\\
 Verification    =         SUCCESSFUL\\
 END OF ADI Benchmark\\

 ADI Benchmark Completed.\\
 Size            =  100 x  200 x  300\\
 Iterations      =                100\\
 Time in seconds =               0.83\\
 Operation type  =   double precision\\
 Verification    =         SUCCESSFUL\\
 END OF ADI Benchmark
\end{minipage}
\newpage
\section{Сравнение скорости работы алгоритмов}
\textbf{Последовательный алгоритм:}\\

Теоретическая оценка: $O(I * nx * ny * nz)$\\

\textbf{Параллельный алгоритм:}\\

Теоретическая оценка: $O(I * max(nx, ny, nz) * ???)$
\end{document}
